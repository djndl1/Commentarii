\documentclass{amsart}
\usepackage[english]{babel}

%%%%%%%%%% Start TeXmacs macros
\newcommand{\tmdfn}[1]{\textbf{#1}}
\newcommand{\tmem}[1]{{\em #1\/}}
\newcommand{\tmop}[1]{\ensuremath{\operatorname{#1}}}
\newcommand{\tmstrong}[1]{\textbf{#1}}
\newtheorem{theorem}{Theorem}
%%%%%%%%%% End TeXmacs macros

\begin{document}

\title{Finite-Dimensional Vector Space}

\author{DJN\_DL}

{\maketitle}

{\tmdfn{Given a list of vectors $v_1, v_2, \ldots, v_n \in V (F)$, and a list
of scalars $c_1, c_2, \ldots, c_n (i = 1, 2, \ldots, n) \in F$
\[ c_1 v_1 + c_2 v_2 + \ldots + c_n v_n \]
is then called {\tmem{linear combination}} of $\{ v_1, v_2, \ldots, v_n \}$.
All linear combinations of a certain list of vectors form a set called
{\tmem{spanning set}}, notation by $\tmop{span} \{ v_1, v_2, \ldots, v_n
\}$.}}

\

{\tmstrong{Span is the smallest containing subspace}}

\begin{theorem}
  The span of a list of vectors is the smallest subspace containing all the
  vectors of the list.
\end{theorem}

\begin{proof}
  For a list of vectors $v_1, v_2, \ldots, v_n \in V$
  
  Let $a_i = 0 \tmop{for} \quad i = 1, 2, \ldots, n$, $a_1 v_1 + a_2 v_2 +
  \ldots + a_n v_n = 0 \in \tmop{span} (v_1, v_2, \ldots, v_n)$
  
  $\{ b_i \} \tmop{for} i = 1, 2, \ldots, n$ is another list of scalars than
  $\{ a_i \}, a_i \neq 0$ here
  \[ \underset{\in \tmop{span}}{(b_1 v_1 + \ldots + b_n v_n)} + \underset{\in
     \tmop{span}}{(a_1 v_1 + a_2 v_2 + \ldots + a_n v_n)} = (a_1 + b_1) v_1 +
     \ldots + (a_n + b_n) v_n \in \tmop{span} \{ v_1, v_2, \ldots, v_n \} \]
  \qquad{\tmem{$\lambda$}} is a scalar in $F$
  \[ \lambda (a_1 v_1 + \ldots + a_n v_n) = \lambda a_1 v_1 + \ldots + \lambda
     a_n v_n \in \tmop{span} \{ v_1, v_2, \ldots, v_n \} \]
  Thus $\tmop{span} \{ v_1, v_2, \ldots, v_n \}$ is a subspace of V, and since
  any subspace containing all vectors of $\{ v_i \}$ must contain their span
  (from the definition of vector space), any subspace other than the span is
  ``larger'' (containing more vectors) than the span.
  
  \ 
\end{proof}

If a list of vectors form a span which is equal to $V$, in other words,
$\tmop{span} \{ v_1, v_2, \ldots, v_n \} = V$, we say $v_1, v_2, \ldots, v_n$
span $V$. If a vector space can be spanned by a finite set of vectors within
itself, it is a {\tmem{finite dimensional vector space}}, otherwise, an
{\tmem{infinite vecotr space.}} (Not really defined yet here)

\

{\tmstrong{Polynomial $P (F)$}}

A function $P : F \rightarrow F$ is said to be a {\tmem{po{\tmem{}}lynomial}}
with coeffecients if there are $a_0, a_1, a_2, \ldots, a_m \in F$ s.t.
\[ p (x) = a_0 + a_1 x + a_2 x^2 + \ldots + a_m x^m \tmop{for} x \in F \]
$P (F)$ is the set of all polynomials with coefficients in $F$, it is a
subspace of $F^F$. Every polynomial can be uniquely determined by its
coefficients.

A polynomial is said to have a {\tmem{degree}} of $m$ if there are
coefficients $a_m \neq 0, a_i (i > m) = 0$ s.t.
\[ p (x) = a_0 + a_1 x + \ldots + a_m x^m + \underset{= 0}{(a_i x^{m + 1} +
   \ldots)} \]


Zero polynomial is defined to have a degree $- \infty$

$P_m (F)$ is defined as the set containing all polynomials of which degree is
no more than $m$.

\

What if a vector in a span can be uniquely represented, i.e. only one list of
scalars makes it a linear combination of a certain list of vectors. Here we
define {\tmem{linear independence}}.

\

Suppose a list of vectors $v_1, v_2, \ldots, v_n \tmop{in} V_F$, if the only
choice to make $a_1 v_1 + a_2 v_2 + \ldots + a_n v_n$ is $a_1 = a_2 = \ldots =
a_n = 0$, then $v_1, v_2, v_3, \ldots, v_n$ are {\tmem{linear independent.}}

\

The definition is just a unique of representatin of zero. Suppose there is a
nonzero vector $v \in \tmop{span} \{ v_1, v_2, \ldots, v_n \}$, and that it
cannot be uniquely represented, that is

there exist $\{ a_i \}$ $\{ b_i \}$ s.t.
\begin{equation}
  v = a_1 v_1 + a_2 v_2 + \ldots + a_n v_n
\end{equation}
\begin{equation}
  v = b_1 v_1 + b_2 v_2 + \ldots + b_n v_n
\end{equation}
$(1) - (2)$: $(a_1 - b_1) v_1 + \ldots + (a_n - b_n) v_n = 0$

Since $\{ v_i \}$ is linearly independent, $a_i - b_i = 0 (i = 1, 2, \ldots,
n) \rightarrow a_i = b_i$, thus $v$ must be uniquely represented by $\{ v_i
\}$.

Conversely, if all vectors can be uniquely represented, choose $0$, all
coefficients must be zero, implying linear independence (the definition of
linear independence). Thus, for a list of vectors, linear independence is
equivalent to unique representation. And {\tmem{linear independence}} is so
defined that we can coefficients other than all zeros s.t. $a_1 v_1 + \ldots .
+ a_n v_n = 0$.

\end{document}
